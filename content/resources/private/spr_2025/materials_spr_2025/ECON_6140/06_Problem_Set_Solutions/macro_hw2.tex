\documentclass[10pt]{article}

\input{/Users/gabesekeres/Dropbox/LaTeX_Docs/pset_preamble.tex}

\course{ECON 6140}
\pset{2}
\begin{document}
\maketitle


\section*{Taxation and human capital}

\begin{enumerate}
	\item The household takes prices, wages, rental prices, interest rates, and taxes as given, and maximizes their utility by choosing a stream of consumption, total labor, productive labor, capital investment, next-period capital, next-period human capital, and bond investment. They solve the problem \[\max_{\{c_t,n_t,n_{m,t},x_t,k_{t+1},h_{t+1},b_{t+1}\}} \sum_{t=0}^\infty \beta^t u(c_t,1-n_t)\]subject to, for $t = 0,1,\dots$, \begin{align*} (1+\tau_t^c)c_t + p_{k,t}x_t + b_{t+1} &\le (1-\tau_t^n)w_t (h_tn_{m,t}) - \tau^w_t + r_tk_t - \tau_t^k(r_t-\delta_k)k_t) + (1+(1-\tau_t^b)r_t^b)b_t \\ k_{t+1} &\le x_t + (1-\delta_k)k_t\\ h_{t+1} &= B(n_t-n_{m,t})^\gamma h_t^{1-\gamma} + (1-\delta_h)h_t \\ 0 &= \lim_{T\to\infty} \beta^T u'_c(c_T,1-n_T)b_{T+1} \\0 &\le (c_t,x_t,k_{t+1}) \\ n_t &\in [0,1]\end{align*}where $\tau_t^c$ is the consumption tax, $\tau_t^n$ is the labor tax, $\tau_t^k$ is the capital rent tax, $\tau_t^b$ is the capital gains tax, and $\tau_t^w$ is the (non-distortionary) lump-sum tax on income. All other variables are as defined, either in the problem or the notes. The price of capital investment, $p_{k,t}$, will end up being 1 in equilibrium, as money can be freely allocated between investment and consumption. For simplicity, we work in terms of $n_{m,t}$ and $n_t$, endogenizing the conditions that $n_{m,t} + n_{h,t} = n_t$ and that $l_t + n_t = 1$. \\ \\ The firm solves the simple, unconstrained problem \[\max_{k_t,z_t} F(k_t,z_t) - r_tk_t - w_tz_t \equiv \max_{k_t,n_{m,t}} F(k_t,h_tn_{m,t}) - r_tk_t - w_t(h_tn_{m,t})\]for each $t = 0,1,\dots$, where the second problem endogenizes the productive effect of human capital.
	\item A recursive competitive equilibrium is a stream of prices $\{w_t\opt,r_t\opt,(r_t^b)\opt\}_{t=0}^\infty$, a stream of allocations $\{c_t\opt,n_t\opt,n_{t,m}\opt,x_t\opt,k_{t+1}\opt,h_{t+1}\opt\}_{t=0}^\infty$, and a sequence of bond holdings $\{b_{t+1}\opt\}$ such that \begin{enumerate} \item Given prices, the allocations and bonds satisfy the household's maximization problem \item Given prices, the allocations satisfy the firm's maximization problem \item Markets clear (meaning that the supply and demand of labor and capital are the same in the household and firm's problems respectively) \item $b_0\opt = b_0 = 0$, $k\opt_0 = k_0 > 0$, $p_{k,0}\opt = p_{k,t}\opt = 1$, and $h_0\opt = h_0$ are each given \end{enumerate}Note that this equilibrium implicitly defines the level of investment in human capital under the model primitives, so the equilibrium formulation is identical to the class formulation, with the addition of $n_{t,m}$ and $h_{t+1}$. Formally, the stream $\{h_t\opt\}_{t=0}^\infty$ is characterized precisely by $h_0$ (given) and condition (a), that the equilibrium allocations are optimal for the household. Additionally, we take the problem parameters (including, importantly, the stream of taxes!) as exogenous.
	\item The household's Lagrangian is \begin{align*} \mathcal{L} &= \sum_{t=0}^\infty \beta^t\Bigg\{ u(c_t,1-n_t) + \\ &+ \lambda_t \Big[ (1-\tau_t^n)w_t (h_tn_{m,t}) - \tau^w_t + r_tk_t - \tau_t^k(r_t-\delta_k)k_t + (1+(1-\tau_t^b)r_t^b)b_t \\ &- (1+\tau_t^c)c_t - p_{k,t}x_t - b_{t+1}\Big] \\ &+ \theta_t [x_t + (1-\delta_k)k_t - k_{t+1}] \\ &+ \mu_t \barl B(n_t-n_{m,t})^\gamma h_t^{1-\gamma} + (1-\delta_h)h_t - h_{t+1}\barr\Bigg\}\end{align*}where the non-negativity constraints and the constraint $n_t \le 1$ can be eliminated because of the Inada conditions on utility. This admits the first order conditions \begin{align*} 0 &= u_c'(c_t,1-n_t) -\lambda_t(1+\tau_t^c) &&(c_t) \\ 0 &= -u'_n(c_t,1-n_t)+\mu_tB\gamma(n_t-n_{m,t})^{\gamma-1}h_t^{1-\gamma} &&(n_t) \\0&=\lambda_t(1-\tau_t^n)w_th_t - \mu_tB\gamma(n_t-n_{m,t})^{\gamma-1}h_t^{1-\gamma} &&(n_{m,t})\\0 &= -\lambda_tp_{k,t} + \theta_t &&(x_t) \\ 0&= -\theta_t + \beta\barl\lambda_{t+1}(r_{t+1} - \tau_{t+1}^k(r_{t+1}-\delta_k)) + \theta_{t+1}(1-\delta_k)\barr &&(k_{t+1})\\ 0 &= -\mu_t + \beta \barl \lambda_{t+1}(1-\tau_{t+1}^n)w_{t+1}n_{m,t+1} + \mu_{t+1}\parl B(1-\gamma) \parl  \frac{n_{t+1}-n_{m,t+1}}{h_{t+1}}\parr^\gamma + (1-\delta_h)\parr\barr &&(h_{t+1}) \\ 0&= -\lambda_t + \beta\lambda_{t+1}(1 + (1-\tau_{t+1}^b)r_{t+1}^b) &&(b_{t+1}) \\ 0 &= \lim_{T\to\infty} \beta^T \lambda_T k_{t+1} &&(\text{TVC}) \\0 &= \lim_{T\to\infty} \beta^T \lambda_T b_{t+1} &&(\text{No Ponzi})\end{align*}for $t = 0,1,\dots$. The first order conditions for the firm are, as always,\[F_k(k_t,h_tn_{m,t}) = r_t \qquad \text{ and }\qquad F_n(k_t,h_tn_{m,t}) = w_th_t\]with these conditions, we can characterize the steady state, assuming that the exogenous streams of taxes $\tau$, bond returns $r^b$, and government spending $g$ are non-stochastic. We have that \begin{align*} p_k[1-\beta(1-\delta_k)] &= \beta [(1-\tau^k)F_k(k\opt,n\opt_m h\opt) + \tau^k\delta_k] &&(k) \\1 &= \beta[1 + \rho] &&(b) \\ u'_n(c\opt,1-n\opt) &= u'_c(c\opt,1-n\opt) \frac{1-\tau^n}{1+\tau^c}F_n(k\opt,h\opt n_m\opt) &&(n_m)\\ (1-\beta(1-\delta_h)) \frac{1}{B\gamma} \parl \frac{n\opt-n_m\opt}{h\opt}\parr^{1-\gamma} &= \beta \parl 1 + \frac{1-\gamma}{\gamma} \frac{n\opt - n_m\opt}{h\opt}\parr &&(h) \\ x\opt &= \delta_k k\opt &&(x) \\ h\opt &=  \parl \frac{B}{\delta_h}\parr^\frac{1}{\gamma}(n\opt - n_m\opt) &&(n) \\ c\opt + \delta_kk\opt + g &= F(k\opt,h\opt n_m\opt) &&(c) \end{align*}where, as in the notes, we define $\rho = (1 + \tau^b)r^b$. We can unpack condition $(n_m)$ to observe the effects of taxes on the steady-state level of human capital, recalling that $F_n(\cdot) = w h\opt$, and observing that no taxes appear explicitly in conditions $(h)$ and $(n)$, the other conditions that concern human capital. We have that \[u'_n(\cdot) = u'_c(\cdot) \frac{1-\tau^n}{1+\tau^c}w h\opt \Longrightarrow h\opt = \frac{u'_n(\cdot)}{u'_c(\cdot)} \frac{1}{w} \frac{1+\tau^c}{1-\tau^n}\]So we can directly see that $h\opt$ is increasing in both $\tau^c$ and $\tau^n$. Both of these conditions make sense -- as consumption and productive labor get more expensive, the household is incentivized to divert energy towards human capital. Additionally, $h\opt$ is not dependent on $\tau^x$ and $\tau^k$, as those two taxes affect only the ratio of investment between bonds and future capital, and it is not dependent on $\tau^w$ as that is a lump-sum tax and therefore non-distortionary.
	\item Describe the changes in the following taxes on the stock of capital, human capital, capital output ratios, and labor: \begin{enumerate} \item Capital gains (in our terms, $\tau^b$). Changes in the capital gains tax $\tau^b$ will have no effect on any variables in the steady state. To see why, note that the only place that the capital gains tax appears in the steady state equations is in equation $(b)$, as a part of $\rho$. Since this condition must hold in steady state, any change in $\tau^b$ will be compensated by a corresponding change in $r^b$, like we saw in class. Formally, we have that \[\frac{\partial\rho}{\partial \tau^b} = \frac{\partial }{\partial \tau^b}\barl (1+\tau^b)r^b\barr = \frac{\partial}{\partial \tau^b} \underbrace{\barl \frac{1}{\beta} - 1\barr}_{\text{constant}} = 0\]Since $\rho$ is necessarily a constant, it will not change when $\tau^b$ changes. Thus, a change in $\tau^b$ will have no effect in equilibrium. We saw in class the same thing, where changes in the capital gains tax are compensated by changes in the risk-free return rate. \item Labor income (in our terms, $\tau^n$). Note first that $\tau^n$ appears only in the steady state condition $n_m$. Holding all else constant, and rearranging, we can see that this condition becomes\[u'_c(c\opt,1-n\opt) = \frac{1}{1-\tau_n} \barl (1+\tau^c)F_n(\cdot) u'_n(c\opt,1-n\opt)\barr\]so $u'_c(\cdot)$ is increasing in $\tau^n$. Since $u$ is increasing and concave in consumption, this means that $c\opt$ is decreasing in $\tau^n$, which implies that $x\opt$ is decreasing in $\tau^n$, since the two move together. Thus, since $x\opt = \delta_kk\opt$, $k\opt$ is also decreasing in $\tau^n$. Human capital $h\opt$ and labor $n\opt$ are not changing in $\tau^n$. To see why, note that both are pinned down by the steady state equations $(n_m)$, $(n)$, and $(h)$. We saw earlier that the change in $\tau^n$ is compensated by a change in $u'_c(\cdot)$, so the other choices in $(n_m)$ are not affected. Other than that, $\tau^n$ does not appear in the other steady state equations characterizing $h\opt$ and $n\opt$, so they do not change when $\tau^n$ changes. For an intuition, observe that changing the bond ratio will change how investment is divided between consumption, capital, and bonds, but will not change how much the household chooses to work (or how they divide that work into productive labor and human capital investment labor). Finally, the capital-output ratio $k\opt / F(k\opt,h\opt n_m\opt)$ will depend on whether the production function faces increasing, decreasing, or constant returns to scale. We know that $h\opt$ and $n_m\opt$ do not change, and we know that $k\opt$ decreases when $\tau^n$ increases. Thus, an increase in $\tau^n$ will lead to an increase in capital-output ratio if there are increasing returns to scale, a decrease if there are decreasing returns to scale, and no change if there are constant returns to scale. The same holds in the opposite direction if $\tau^n$ decreases. \end{enumerate}
	\item If the labor income tax $\tau^n$ changes by 10\%, we could replicate the changes described in (4b) by changing $\tau^c$ such that the ratio $\frac{1-\tau^n}{1+\tau^c}$ is the same as when $\tau^n$ changed. To see why this replicates the change precisely, note that the only place that either $\tau^n$ or $\tau^c$ appear in the steady state is in that precise ratio. As long as the ratio remains constant, there will be no effect on the steady state values. Specifically, a decrease in $\tau^n$ could be replicated by a decrease in $\tau^c$. Intuitively, it must be the case that the marginal utility of leisure and the marginal utility of consumption are in ratio of the respective taxes, so we could replicate the same choices of leisure and consumption by either increasing the tax on working or increasing the tax on consumption -- either way, the household will move to increase leisure. The same holds, of course, for decreases.
\end{enumerate}




\section*{A primer to OLG}

\begin{enumerate}
	\item Each generation $t$ solves the problem \[\max_{s_t} \ln(c^t_t) + \beta \ln(c^t_{t+1})\]subject to \begin{align*} c_t^t &\le w_t - \delta_{y,t} - s_t \\ c_{t+1}^t &\le (1+r_{t+1})s_t - \delta_{o,t+1} \end{align*}(which will, of course, each hold with equality). The Lagrangian, substituting the constraints, is \[\mathcal{L} = \ln(w_t - \delta_{y,t} - s_t) + \beta \ln((1+r_{t+1})s_t - \delta_{o,t+1})\]which admits the first order condition\[\frac{\partial \mathcal{L}}{\partial s_t} = -\frac{1}{w_t - \delta_{y,t}-s_t} + \frac{\beta(1+r_{t+1})}{(1+r_{t+1})s_t - \delta_{o,t+1}} = 0\]implying that \[s_t\opt = \frac{\beta}{1+\beta}(w_t - \delta_{y,t}) + \frac{1}{1+\beta}\frac{\delta_{o,t+1}}{1+r_{t+1}}\]Substituting back to get optimal consumption, we get that \begin{align*} (c_t^t)\opt &= w_t - \delta_{y,t} - \parl \frac{\beta}{1+\beta}(w_t - \delta_{y,t}) + \frac{1}{1+\beta}\frac{\delta_{o,t+1}}{1+r_{t+1}}\parr \\ &= \frac{1}{1+\beta}\parl w_t - \delta_{y,t} - \frac{\delta_{o,t+1}}{1+r_{t+1}}\parr \\ (c_{t+1}^t)\opt &= (1+r_{t+1})\parl \frac{\beta}{1+\beta}(w_t - \delta_{y,t}) + \frac{1}{1+\beta}\frac{\delta_{o,t+1}}{1+r_{t+1}}\parr - \delta_{o,t+1} \\ &= \frac{\beta}{1+\beta}\parl (1+r_{t+1})(w_t-\delta_{y,t}) - \delta_{o,t+1}\parr \end{align*}For the initial old generation, we have that they supply no labor and receive no wage, so they have no real choice problem -- they consume their endowment, along with any tax imposed by the government. Formally,\[c^0_0 = (1+r_0)A_0 - \delta_{o,0}\]Note that the government's budget constraint and spending do not factor in here at all -- the households take them as exogenous, and optimize savings relative to the taxes they see. For consumption to be positive, it's necessary that the taxes are not too high. Specifically, we need that $\delta_{y,t} < w_t$, and that $\delta_{o,t+1}$ is not too large. It would suffice to impose that $\delta_{o,t+1} \le 0$, so the government is on net making transfers to the old generation, as we would expect from actual policy.
	\item In the steady state with Cobb-Douglas production technology and constant labor supply of 1, we can precisely characterize the wage and interest rate:\[w = \frac{\partial F(k,1)}{\partial n} = (1-\alpha)k^\alpha \qquad \text{ and } \qquad r = \frac{\partial F(k,1)}{\partial k} - 1 = \alpha k^{\alpha -1}-1\]From part 1, we have that the steady state optimal savings rate is \[s = \frac{\beta}{1+\beta} (w - \delta_y) + \frac{1}{1+\beta} \frac{\delta_o}{1+r}\]Additionally, the government must fund in each period $g$ and the interest on their debt $rd$, so their budget constraint is \[g + rd = \delta_y + \delta_o\]Finally, the market for assets must clear, meaning that the young must save precisely enough to buy the capital and debt from the old. We must have that \[s = k + d\]Along with the exogenous parameters, these conditions precisely characterize the steady state.
	\item Observe that the condition $s = k+d$ characterizes precisely where the savings of the economy go. If $d > 0$, then some savings must go to debt rather than capital. As $d$ increases, $r$ also increases, assuming that $\alpha \in (0,1)$. This would necessarily lower wages $w$ and consumption $c$ in the steady state, at all levels of $d$. Thus, since the economy starts with $d_0 = 0$, optimally the government will hold no debt. Relating to what we talked about in class, we can see that this economy does \emph{not} face dynamic inefficiencies, since $r > 0$ and there is no population growth. In that case, it cannot be that $k\opt > k^{\text{gold}}$, so there are no gains to reducing capital. Thus, the government should hold no debt and finance their spending purely through (non-distortionary) taxes.
	\item I solved for the steady state in Julia, making some additional assumptions for tractability. Along with assuming that $d = 0$, I also imposed that $\delta_o = 0$, so there are no taxes (or transfers) on the old. Since taxes are linear in the government's budget constraint and the households take $r$ as given, this does not impose any behavioral constraints on the problem, and ensures a single (interior) solution to the government's optimal policy, rather than the continuum when both are free variables. I used the \texttt{NLsolve} root-finding algorithm, and the entirety of the code used for this problem is below.\footnote{This is a slow algorithm, and it could be improved if I used Newton-Raphson since the policy function is smooth. However, I am lazy and don't want to write an entire gradient descent algorithm. This is fast enough for this problem, it runs entirely in $0.25$ seconds.} I got the steady state levels:\begin{align*}k &= 0.16576237075421454 \\ F(k,1) &= 0.5832379634630681 \\ w &= 0.40826657442414765 \\ r &= 0.0555555416033509 \\ g &= 0.058323796346306814 \\ \delta_y &= 0.058323796346306814 \\ \delta_o &= 0 \end{align*}Checking optimal savings policy to confirm, I got that it equalled the steady-state capital, which would be a confirmation that this algorithm worked. From these, optimal consumption levels are \[c_y = 0.18418040951465306 \quad ; \quad c_0 = 0.17497138672616994\]
	\item The immediate effect of the government/output ratio changing to 0.15 will be that $g$ increases, leading to an increase in $\delta_y$, as it is directly in proportion to the government policy. We can directly calculate the new steady state and compare: \[ \begin{array}{|c|c|c|c|} \hline \text{Variable} &\text{Old Steady State} & \text{New Steady State} & \text{Change} \\\hline k & 0.1658 & 0.1464 & \downarrow \\\hline F(k,1) & 0.5832 & 0.5619& \downarrow \\\hline w & 0.4083 & 0.3933& \downarrow \\\hline r & 0.0556 & 0.1515& \uparrow \\\hline g & 0.0583 & 0.0843& \uparrow \\\hline \delta_y & 0.0583 & 0.0843& \uparrow \\\hline \delta_o & 0 & 0 & =\\\hline c_y & 0.1842 & 0.1627 & \downarrow \\\hline c_o & 0.1750 & 0.1686 & \downarrow \\\hline \end{array}\]As we can see, the shift decreases steady-state consumption, output, capital, and wages, and increases the interest rate, government spending, and taxes on the young.
	\item If taxes are now labor-income taxes rather than lump-sum, the generation's problem becomes \[\max_{s_t} \ln(c_t^t) + \beta \ln(c_{t+1}^t) \]subject to\begin{align*} c_t^t &\le w_t(1-\tau_t) - s_t \\ c_{t+1}^t &\le s_t(1 + r_{t+1})\end{align*}The Lagrangian is now \[\mathcal{L} = \ln(w_t(1-\tau_t) - s_t) + \beta \ln (s_t(1 + r_{t+1}))\]which admits the first order condition\[\frac{\partial \mathcal{L}}{\partial s_t} = -\frac{1}{w_t(1-\tau_t) - s_t} + \frac{\beta }{s_t} = 0 \Longrightarrow s_t\opt = \frac{\beta}{1+\beta}w_t(1-\tau_t) \]and substituting back to get optimal consumption, we get that \begin{align*} (c_t^t)\opt &= \frac{1}{1+\beta}w_t(1-\tau_t) \\ (c^t_{t+1})\opt &= \frac{\beta}{1+\beta} w_t(1-\tau_t)(1+r_{t+1})\end{align*}In the steady state, the wage and interest rate remain the same, as the technology has not changed:\[w = \frac{\partial F(k,1)}{\partial n} = (1-\alpha)k^\alpha \qquad \text{ and } \qquad r = \frac{\partial F(k,1)}{\partial k} - 1 = \alpha k^{\alpha -1}-1\]The steady state optimal savings rate is simply\[s = \frac{\beta}{1+\beta}w(1-\tau)\]and the government's budget constraint is now \[g+rd = w\tau \Longrightarrow \tau = \frac{g}{w}\]Finally, the market for assets must again clear so we need \[s = k + d\]Computationally, we have that using the parameters from above\begin{align*} k &= 0.16576237075421468 \\ F(k,1) &= 0.5832379634630682 \\ w &= 0.40826657442414765 \\ r &= 0.0555555416033502 \\ g &= 0.058323796346306814 \\ \tau &= 0.14285714285714288 \end{align*}Checking optimal savings policy to confirm, I got that it equalled the steady-state capital, which would be a confirmation that this algorithm worked. From these, optimal consumption levels are \[c_y = 0.18418040951465306 \quad ; \quad c_0 = 0.17497138672616994\]So nothing has changed from the steady state above, except for the tax levels!
	\item If the government-output ratio increases to $0.15$, we can calculate the new steady state and compare, like above:\[ \begin{array}{|c|c|c|c|c|} \hline \text{Variable} &\text{Old Steady State} & \text{New Steady State} & \text{Change} & \text{Same?} \\\hline k & 0.1658 & 0.1464 & \downarrow & \checkmark \\\hline F(k,1) & 0.5832 & 0.5619& \downarrow & \checkmark \\\hline w & 0.4083 & 0.3933& \downarrow& \checkmark \\\hline r & 0.0556 & 0.1515& \uparrow& \checkmark \\\hline g & 0.0583 & 0.0843& \uparrow& \checkmark \\\hline \tau & 0.1429 & 0.2142& \uparrow & \text{NA} \\\hline c_y & 0.1842 & 0.1627 & \downarrow & \checkmark \\\hline c_o & 0.1750 & 0.1686 & \downarrow & \checkmark\\\hline \end{array}\]So everything is the same as above! This is a strange result, but comes from the fact that in the lump-sum model, taxes were only on the young, and in this model taxes are only on those making income -- basically, the young. While we are rescaling the taxes, an increase in them will affect the generations in the exact same way.
\end{enumerate}



\paragraph{Code.} I wrote the code in Julia, and it is included here:

\lstinputlisting[language=Julia]{./macro_hw2_code/main.jl}


















\end{document}