\documentclass[10pt]{article}

\input{/Users/gabesekeres/Dropbox/LaTeX_Docs/pset_preamble.tex}

\course{ECON 6140}
\pset{5}
\begin{document}
\maketitle

\section*{Credit Markets and Economic Growth, continued}


\begin{enumerate}
	\item You can not necessarily compute the competitive equilibrium by solving the planner's problem, as the First Welfare Theorem requires that markets are complete, which we do not have once the credit markets are shut down. Intuitively, the solution to the planner's problem and the solution to the competitive equilibrium will coincide only if the planner's problem involves no reassignment of capital between workers. If there is any, which is (at least somewhat) likely, then the solutions will be different.
	\item Fixing $k$ and $w$, the entrepreneur $i$ solves the problem\[\max_l f(e_i,k,l) - w\cdot l \equiv \max_l e_i^\nu l^{(1-\nu)(1-\alpha)} k^{\alpha(1-\nu)} - w \cdot l\]which admits first order condition\[0 = (1-\nu)(1-\alpha)e_i^\nu l^{(1-\nu)(1-\alpha) - 1} k^{\alpha(1-\nu)} - w\]so \[ l^{(1-\nu)(1-\alpha) - 1} = \frac{w}{(1-\nu)(1-\alpha)e_i^\nu k^{\alpha(1-\nu)}} \Longrightarrow  l\opt = \parl \frac{w}{(1-\nu)(1-\alpha)e_i^\nu k^{\alpha(1-\nu)}}\parr^{\frac{1}{(1-\nu)(1-\alpha)-1}}\]
	\item Given a certain $k_t$ and $w_t$, the entrepreneur $i$ earns profits of \[f(e_i,k_t,l\opt(w_t,k_t)) - w_t \cdot l\opt(w_t,k_t)\]Notice first that from the first order condition, we have that $f(e_i,k_t,l\opt(w_t,k_t)) = \frac{w_t\cdot l\opt(w_t,k_t)}{(1-\alpha)(1-\nu)}$ so profit becomes\[\frac{w\cdot l\opt(w_t,k_t)}{(1-\alpha)(1-\nu)} - w_t\cdot l\opt(w_t,k_t) = w_t\cdot l\opt(w_t,k_t)\barl \frac{1}{(1-\alpha)(1-\nu)}-1\barr\]so inputting the optimal labor demand, we get that profits are\[w_t  \parl \frac{w_t}{(1-\nu)(1-\alpha)e_i^\nu k_t^{\alpha(1-\nu)}}\parr^{\frac{1}{(1-\nu)(1-\alpha)-1}}\barl \frac{\alpha + \nu - \alpha\nu}{(1-\alpha)(1-\nu)}\barr\]which, in a nicer form for later, is \[w_t \parl \frac{(1-\nu)(1-\alpha)e_i^\nu k_t^{\alpha(1-\nu)}}{w_t}\parr^{\frac{1}{\alpha+\nu-\alpha\nu}} \barl \frac{\alpha + \nu - \alpha\nu}{(1-\alpha)(1-\nu)}\barr\]
	\item The market clearing wage is that where the total labor supply $\pi$ matches the total labor demand $l\opt(w_t,K_t)$ (from only the high-ability entrepreneurs). Specifically, we need to find $w$ that solves\[\parl \frac{w}{(1-\nu)(1-\alpha)e_H^\nu K_t^{\alpha(1-\nu)}}\parr^{\frac{1}{(1-\nu)(1-\alpha)-1}} = \pi\Longrightarrow \pi = \parl \frac{(1-\nu)(1-\alpha)e_H^\nu K_t^{\alpha(1-\nu)}}{w}\parr^{\frac{1}{\alpha + \nu - \alpha \nu}}\]so we get that\[w\opt = \frac{(1-\nu)(1-\alpha)e_H^\nu K_t^{\alpha(1-\nu)}}{\pi^{\alpha+\nu-\alpha\nu}}\]
	\item Substituting the market clearing wage into our expression for profit from part (3), we get that it becomes\[\parl \frac{(1-\nu)(1-\alpha)e_H^\nu K_t^{\alpha(1-\nu)}}{\pi^{\alpha+\nu-\alpha\nu}}\parr \parl \frac{\pi^{\alpha + \nu - \alpha\nu}k_t^{\alpha(1-\nu)}}{K_t^{\alpha(1-\nu)}}\parr^{\frac{1}{\alpha + \nu - \alpha\nu}} \barl \frac{\alpha + \nu - \alpha\nu}{(1-\nu)(1-\alpha)}\barr\]Simplifying, this becomes\[(\alpha + \nu - \alpha\nu)e_H^\nu \pi^{1 - \alpha - \nu + \alpha\nu}K_t^{\alpha(1-\nu) - \frac{\alpha(1-\nu)}{\alpha + \nu - \alpha\nu}}k_t^{\frac{\alpha(1-\nu)}{\alpha + \nu -\alpha\nu}}\]and since $1-\alpha - \nu + \alpha\nu = (1-\alpha)(1-\nu)$, and $\alpha(1-\nu) - \frac{\alpha(1-\nu)}{\alpha + \nu - \alpha\nu} = \frac{-\alpha(1-\alpha)(1-\nu)^2}{\alpha+\nu-\alpha\nu}$, we get that profit can be expressed as $AK_tk_t^\phi$, where \[AK_t = (\alpha + \nu - \alpha\nu)e_H^\nu \pi^{(1-\alpha)(1-\nu)}K_t^{\frac{-\alpha(1-\alpha)(1-\nu)^2}{\alpha+\nu-\alpha\nu}}\qquad \text{ and } \phi = \frac{\alpha(1-\nu)}{\alpha + \nu - \alpha\nu}\]
	\item The (present-value) Hamiltonian is\[\mathcal{H}(t) = \exp(-\rho t) \frac{c(t)^{1-\sigma}}{1-\sigma}dt + \lambda(t)\barl AK_tk(t)^\phi - \delta_t k(t) - c(t)\barr\]which admits first order conditions \begin{align*} 0 &= \exp(-\rho t)c(t)^{-\sigma} - \lambda(t) &&(c) \\-\dot{\lambda}(t) &= \lambda(t) \barl AK_t \phi k(t)^{\phi - 1} - \delta_t\barr &&(k)\end{align*}So combining the two conditions (along with the time derivative of the first), we get the classic Euler equation for consumption:\[\frac{\dot{c}(t)}{c(t)} = \frac{1}{\sigma}\barl AK_t \phi k(t)^{\phi-1} - \delta_t - \rho\barr\]and the other ODE is our law of motion for capital: \[\dot{k}(t) = AK_t k(t)^{\phi} - \delta_t k(t) - c(t)\]Along with initial conditions, these perfectly characterize the balanced growth path of the economy. 
	\item If $k_t = K_t$, our budget constraint becomes\[\dot{K}(t) = AK(t)^{\phi+1} - \delta_t K(t) - c(t)\]so the (present-value) Hamiltonian is\[\mathcal{H}(t) = \exp(-\rho t) \frac{c(t)^{1-\sigma}}{1-\sigma}dt + \lambda(t)\barl AK(t)^{\phi+1} - \delta_t K(t) - c(t)\barr\]which admits first order conditions \begin{align*} 0 &= \exp(-\rho t)c(t)^{-\sigma} - \lambda(t) &&(c) \\-\dot{\lambda}(t) &= \lambda(t) \barl A(\phi+1)K(t)^\phi - \delta_t\barr &&(K)\end{align*}So again combining the two conditions and differentiating the first with respect to $t$, we get the Euler equation for consumption:\[\frac{\dot{c}(t)}{c(t)} = \frac{1}{\sigma} \barl A(\phi+1)K(t)^\phi - \delta_t - \rho\barr\]and the other ODE is the law of motion for capital:\[\dot{K}(t) = AK(t)^{\phi+1} - \delta_t K(t) - c(t)\]
	\item If the initial capital stock belongs to the low-ability agents, there will be no production at all. Since the high-ability agents have $k_t = 0$ and the low-ability agents have $e_L = 0$, and there is no mechanism for transferring capital between agent types, there is no output.
\end{enumerate}

\paragraph{Extra Points Questions.}

\begin{enumerate}
	\item Recall that in the previous homework, the equilibrium dynamics were given by the two ODEs: \begin{align*} \dot{C}(t) &= \frac{C(t)}{\sigma}\parl \alpha(1-\nu)e_H^\nu \pi^{(1-\nu)(1-\alpha)}K(t)^{\alpha(1-\nu)-1}-\delta - \rho\parr &&\coloneqq F(C,K)\\ \dot{K}(t) &= e^\nu_H \pi^{(1-\nu)(1-\alpha)}K(t)^{\alpha(1-\nu)} - \delta K(t) - C(t)&&\coloneqq G(C,K) \end{align*}In the steady state, $\dot{C}(t) = \dot{K}(t) = 0$, which implies that \begin{align*} 0 &= \alpha(1-\nu)e_H^\nu \pi^{(1-\nu)(1-\alpha)}K^{\alpha(1-\nu)-1}-\delta - \rho \\ 0 &= e^\nu_H \pi^{(1-\nu)(1-\alpha)}K^{\alpha(1-\nu)} - \delta K - C\end{align*}Call the (precisely determined) solutions to these equations $C\opt$ and $K\opt$ the steady state values. As we learned from Ryan, we can approximate small deviations around the steady state, where $\hat{C}(t) = C(t) - C\opt$ and $\hat{K}(t) =K(t)-K\opt$, by the Jacobian of the ODE equations. We have that\[\matrixc{\hat{\dot{C}}(t) \\ \hat{\dot{K}}(t)} \approx \matrixc{\partial F(C\opt,K\opt) / \partial C & \partial F(C\opt,K\opt) / \partial K\\ \partial G(C\opt,K\opt) / \partial C & \partial G(C\opt,K\opt) / \partial K}\cdot \matrixc{\hat{C}(t)	\\\hat{K}(t)}\]where the Jacobian matrix at the steady state is\[\matrixc{\frac{1}{\sigma}\parl \alpha(1-\nu)e_H^\nu \pi^{(1-\nu)(1-\alpha)}(K\opt)^{\alpha(1-\nu)-1}-\delta - \rho\parr  & \frac{C\opt}{\sigma}\parl  (\alpha(1-\nu)-1)\alpha(1-\nu)e_H^\nu \pi^{(1-\nu)(1-\alpha)}(K\opt)^{\alpha(1-\nu)-2}\parr \\\\ -1 & \alpha(1-\nu)e_H^\nu \pi^{(1-\nu)(1-\alpha)}(K\opt)^{\alpha(1-\nu)-1} - \delta}\]
	\item The matrix form is as above. I cannot go further in characterizing the local stability of the system. While that answer may get no points, I feel that it is technically correct and I should be given something for at least being funny. Anyways, thanks for reading. Here's a link to a \href{https://www.youtube.com/watch?v=x10vL6_47Dw}{bird livestream}.
\end{enumerate}

\newpage

\section*{Optimal R\&D in the AK World}

\begin{enumerate}
	\item The planner's problem is, similarly to in class, \[\max \int_0^\infty \exp(-\rho t) \frac{c(t)^{1-\sigma}}{1-\sigma}dt\]subject to\begin{align*} c(t) + X(t) &= L(t)^{1-\alpha} \parl \int_0^{A(t)} x_i(t)^{1-\mu}di\parr^{\frac{\alpha}{1-\mu}} \\x_i(t) &= al_i(t) \\ \dot{A}(t) &= bX(t) \\1 &= L(t) + \int_0^{A(t)}l_i(t)di \end{align*}Since $\mu \in (0,1)$, the planner optimally sets $x_i(t) = x(t)$ for all $i$, and therefore also sets $l_i(t) = l(t)$. Feasibility requires that $l(t) = \frac{1-L(t)}{A(t)}$.
	\item With the optimal factor allocations, this problem becomes\[\max \int_0^\infty  \exp(-\rho t) \frac{c(t)^{1-\sigma}}{1-\sigma}dt\]subject to\begin{align*} c(t) + X(t) &= a^\alpha L(t)^{1-\alpha}(1-L(t))^\alpha A(t)^{\frac{\alpha\mu}{1-\mu}} \\ \dot{A}(t) &= bX(t)\end{align*}Once these have been imposed, the only evolving state variable is $A(t)$. The control variables at any time $t$ are $c(t)$ and $L(t)$, and all other variables are endogenously determined precisely. This problem does have a recursive representation, and by combining the laws of motion we can write the Hamilton-Jacobi-Bellman equation \[\rho V(A) = \max_{c,L} \barl \frac{c^{1-\sigma}}{1-\sigma} + V'(A) b \barl a^\alpha L^{1-\alpha}(1-L)^\alpha A^{\frac{\alpha\mu}{1-\mu}} - c\barr\barr\]This model reduces to an AK framework when the production function is linear in $A$, which happens if and only if \[\alpha\mu = 1-\mu\]
	\item Since we have that $\dot{A} = bX$, when we have $A$ growing at a constant rate $g_A$, we have that \[g_A = \frac{\dot{A}}{A} = b\frac{X}{A}\]so the growth rate of the available intermediate inputs is entirely dependent on the R\&D investment level. Essentially, the planner uses a fraction of output to keep the available intermediate goods growing over time.
	\item Intermediate goods producers, who act as monopolists, solve the problem \[\max_{p_i(t)} p_i(t)x_i(t) - w(t) \frac{x_i(t)}{a}\]Since the producers are monopolists, they face isoelastic demand, where $x_i(t) = Bp_i(t)^{-\frac{1}{\mu}}$, where $B$ is some positive parameter and $\frac{1}{\mu}$ is the elasticity of substitution across intermediate goods. The problem then reduces to\[\max_{p_i(t)} p_i(t) Bp_i(t)^{-\frac{1}{\mu}} - \frac{w(t)}{a} Bp_i(t)^{-\frac{1}{\mu}}\]which admits first order condition\[Bp_i(t)^{-\frac{1}{\mu}}\barl \frac{\mu-1}{\mu} + \frac{w(t)}{a\mu} p_i(t)^{-1}\barr = 0 \Longrightarrow p_i(t) = \frac{w(t)}{a} \cdot \frac{1}{1-\mu}\]which, since $\mu \in (0,1)$ and the marginal cost of production is $\frac{w(t)}{a}$, is a constant markup over marginal cost. This comes from (i) the intermediate goods producers being monopolists, and (ii) there being a constant elasticity of substitution between intermediate goods.
	\item Under the AK conditions, where $a\mu = 1-\mu$, the technology is linear in $A$. The balanced growth path in the competitive equilibrium is still the same linear function of $X$ that it is in the social planner problem. However, since there is a markup over marginal cost, final goods producers in the competitive equilibrium will invest less than the social planner will, so R\&D investment along the balanced growth path will be relatively lower than the efficient level.
	\item If the government provided a subsidy to R\&D investment, that would deal with one of the inefficiencies (the fact that final goods producers take $A(t)$ as given, and do not internalize it), but would not deal with the fact the intermediate goods producers also price above marginal cost. To close the gap, a subsidy to R\&D investment (to internalize $A(t)$) could be combined with a production subsidy for intermediate goods, that would compel intermediate goods producers to price at precisely the marginal cost.
\end{enumerate}












\end{document}