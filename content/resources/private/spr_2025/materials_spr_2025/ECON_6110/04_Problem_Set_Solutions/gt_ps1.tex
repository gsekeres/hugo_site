\documentclass[10pt]{article}

\input{/Users/gabesekeres/Dropbox/LaTeX_Docs/pset_preamble.tex}

\course{ECON 6110}
\pset{1}
\begin{document}
\maketitle

\subsection*{Problem 1}

\begin{enumerate}
	\item \begin{proof} Take some profile $s$ that is a Nash equilibrium of the primal game. Note that the set of strategies available in the game in which strategies have been removed by iterated strict dominance is a subset of the set of strategies available in the primal game. Thus, it suffices to show that $s$ is available in the game in which strategies have been removed by iterated strict dominance to show that $s$ is a Nash equilibrium of that game. FSOC, assume not, meaning that there is some player $i$ for which $s_i$ is dominated, meaning that there is some (possibly mixed) strategy $\sigma_i$ such that \[U_i(\sigma_i,s_{-i}) > U_i(s_i,s_{-i})\]Then we can immediately see that $s$ is not a Nash equilibrium of the primal game, since player $i$ would prefer $\sigma_i$ to it. This is a contradiction of the assumption. \end{proof}
	\item \begin{proof} Proof by contrapositive. Take some profile $s$ that is not a Nash equilibrium of the primal game. It remains to show that it is not a Nash equilibrium of the game in which strategies have been removed by iterated strict dominance. Since $s$ is not a Nash equilibrium, using the negation of the Nash equilibrium definition there must exist some player $i$ and some (possibly mixed) strategy $\sigma_i'$ for which \[U_i(\sigma_i',s_{-i}) > U_i(s_i,s_{-i})\]Thus, $s_i$ is strictly dominated for player $i$, so it is not available in the game in which strategies have been removed by iterated strict dominance, so $s$ is not an equilibrium of that game. \end{proof}
	\item \begin{proof} Take some profile $s$ that is a Nash equilibrium of the game where strategies have been removed by iterated weak dominance. FSOC, assume that it is not a Nash equilibrium of the larger game, meaning that for some $i$ there is some (possibly mixed) strategy $\sigma_i$ that they prefer, such that \[U_i(\sigma_i,s_{-i}) > U_i(s_i,s_{-i})\]Then $s_i$ is strictly dominated, meaning that it is weakly dominated, meaning that it would not be available in the game where strategies have been removed by iterated weak dominance. This is a contradiction of the fact that $s$ is a Nash equilibrium of that game. \end{proof} 
	\item Consider the following example: \begin{center} \begin{tabular}{c|cc} & L & R \\\hline U & $1,1$ & $0,0$ \\D & $1,0$ & $-1,0$\end{tabular}\end{center}$D$ is weakly dominated, but $D,L$ is a Nash equilibrium.
\end{enumerate}

\subsection*{Problem 2}

\begin{enumerate}
	\item There is one pure-strategy Nash equilibrium, $(U,L)$. There are a continuum of mixed strategy equilibria. Assume that column chooses $L$ with probability $p$. Then we must have that \[u(U,p) = 4p + 3(1-p) = 4p + 0(1-p) = u(D,p) \Longrightarrow p = 1\]Thus, if column chooses $L$ all the time, row is indifferent. Assume that row chooses $U$ with probability $q$. Column will be indifferent if and only if \[u(R,q) = u(L,q) \Longrightarrow q + (1-q)6= 5q \Longrightarrow q =  \frac{6}{10}\]So there are a continuum of mixed strategy equilibria, where column plays $L$ with probability 1 and row plays $U$ with probability $q \in [0.6,1]$.
	\item There is only one pure-strategy Nash equilibrium, $(U,R)$. Observe that $D$ is strictly dominated by $U$, and once $D$ is eliminated $L$ is strictly dominated by $R$. Thus, this is the only Nash equilibrium.
	\item There are two pure-strategy Nash equilibria, $(U,L)$ and $(M,C)$. There are also mixed strategy equilibria. Assume that column chooses $L$ with probability $p_1$, $C$ with probability $p_2$, and $R$ with probability $p_3$. The expected payoff for each pure strategy for row is \begin{align*} u(U,p) &= 6p_1 + p_2 + 3p_3 \\ u(M,p) &= 2p_1 + 4p_2 + 4p_3 \\ u(D,p) &= 3p_1 + 2p_2 + 3p_3\end{align*}Assume that row chooses $U$ with probability $q_1$, $M$ with probability $q_2$, and $D$ with probability $q_3$. The expected payoff for each pure strategy for column is \begin{align*} u(L,q) &= 6q_1 + 2q_2 + 3q_3 \\ u(C,q) &= q_1 + 7q_2 + 3q_3 \\ u(R,q) &= 4q_1 + 5q_2 + 9q_3\end{align*} By solving the systems $u(U,p) = u(M,p) = u(D,p)$ and $u(L,q) = u(C,q) = u(R,q)$ and verifying that they have no solutions (in the open unit interval), we can confirm that neither player is ever purely mixing. Next, we check whether they will ever mix between two strategies. Assume first that $p_1 = 0$. Then $M$ strictly dominates either $U$ or $D$. If $p_2 = 0$, then $U$ strictly dominates $D$ for any positive $p_1$. Once $D$ is eliminated, $R$ is strictly dominated by a combination of $L$ and $C$, so it will also happen that $p_3 = 0$, and we are back to pure. Thus, it must be the case that $p_3 = 0$, so column mixes between $L$ and $C$. Similarly, assume first that $q_1 = 0$. Then $L$ is strictly dominated, and once $L$ is eliminated we saw above that $q_3 = 0$ also and we are back to pure. Next, assume that $q_2 = 0$. Then $C$ is strictly dominated, and again we are back in the above case where $p_2 = 0$, so it would again happen that $q_3 = 0$. Thus, it must be the case that $q_3 = 0$, and row mixes between $U$ and $M$. \\\\Once we are in the world where row mixes between $U$ and $M$ and column mixes between $L$ and $C$, say that column chooses $L$ with probability $p$ and row chooses $U$ with probability $q$. It must hold that \[u(U,p) = u(M,p) \Longrightarrow 6p + (1-p) = 2p + 4(1-p)  \Longrightarrow p = \frac{3}{7}\]and \[u(L,q) = u(C,q) \Longrightarrow 6q + 1-q = 2q + 7(1-q) \Longrightarrow q = \frac{3}{5}\]So there is one mixed Nash equilibrium, \[\parl \frac{3}{5}U + \frac{2}{5}M,\frac{3}{7}L + \frac{4}{7}C\parr\]\begin{remark} There are a few way easier ways to do this. It would be significantly more straightforward to simply observe that $D$ and $R$ are strictly dominated by some mixed combination of $U,M$ and $L,C$ respectively, which follows directly from the only pure Nash equilibria being in those strategies. I'm choosing to leave my high effort version because of my intense moral fibre that does not allow me to take credit for \href{https://people.as.cornell.edu/sites/default/files/styles/person_image/public/2024-10/ye.jpg}{Finn}'s observations, and also because I already did the work and I'm quite lazy. Hi Ruqing! Hope you're having a good day.\end{remark}
\end{enumerate}

\subsection*{Problem 3}

\begin{enumerate}
	\item We will define the easiest possible probability space $\Omega = \{a,b,c,d\}$, with $\pi(a) = \pi(b) = \pi(c) = \frac{1}{3}$ and $\pi(d)=0$. This state space coincides with the action space $\{(U,L),(U,R), (D,L),(D,R)\}$, which we can do because of the Proposition presented by Ruqing in her week 2 TA section, in which she showed it is without loss to do so. Intuitively, if the state space was larger than the action space, it would necessarily collapse down to the action space. If it were smaller, it could be made equivalent by putting 0 probability on all actions not represented in the state space.
	\item Row has the partitions $P_1 = \{a\}$, $P_2 = \{b,c\}$, and $P_3 = \{d\}$. Column has the partitions $Q_1 = \{a,b\}$, $Q_2 = \{c\}$, and $Q_3 = \{d\}$. Observe that $P_3$ and $Q_3$ have probability 0 of occurring, so we will henceforth ignore them.
	\item $\sigma(P_1) = D$, $\sigma(P_2) = U$, $\sigma(Q_1) = L$, $\sigma(Q_2) = R$. Using the probabilities defined above, we have that $\prob(U,L) = 1/3$, $\prob(U,L) = 1/3$, $\prob(D,L) = 1/3$, and $\prob(D,R) = 0$.
	\item These strategies are a correlated equilibrium. Observe that if row sees $P_2$, they evaluate the conditional probability $\prob(b \mid P_2) = \prob(c \mid P_2) = 1/2$, so their expected utility for playing $U$ is $2 \cdot (1/2) + 0 = 1$, and their expected utility for playing $D$ is $1 \cdot (1/2) + 0 = 1/2$, so they will adhere to the prescribed strategy and play $U$. If row sees $P_1$, they will either $1$ for playing $D$ or 0 for playing $U$, so they will play $D$. Since the game is symmetric, the same holds for column -- if they see $Q_1$, they will perceive a higher expected utility from $L$ than $R$, and if they see $Q_2$ they know they will attain 1 rather than 0. Thus, neither player will deviate and this is a correlated equilibrium.
\end{enumerate}


\subsection*{Problem 4}

\begin{enumerate}
	\item Any policy $a_i \in (0,1)$ is rationalizable. Take some $a_{-i} \in [0,1]$. If $F(a_{-i}) < 1/2$, then $\exists \varepsilon > 0$ such that $F(a_{-i} + \varepsilon) = 1/2$, and thus any $a_i \in (a_{-i},a_{-i} + 2\varepsilon)$ will attain 1, and is a best response. If $F(a_{-i}) > 1/2$, then $\exists \varepsilon > 0$ such that $F(a_{-i} - \varepsilon) = 1/2$, and thus any $a_i \in (a_{-i} - 2\varepsilon,a_{-i})$ will attain 1, and is a best response. Finally, if $F(a_{-i}) = 1/2$, then the only best response is $a_i = a_{-i}$. Thus, by perturbing $a_{-i}$, we can see that any $a_i$ is the best response to some $a_{-i}$, with the only exceptions being $a_i \in \{0,1\}$, which are strictly dominated by $\varepsilon$ and $1-\varepsilon$ for small $\varepsilon$.
	\item Implicitly define the pure Nash equilibrium $a\opt$ such that $F(a\opt) = 1/2$. Observe that this $a\opt$ will exist by the Intermediate Value Theorem. Then note that the payoff for $a\opt$ is \[v_i(a_i\opt,a_{-i}) = \begin{cases} 1 & \text{if } a_{-i} > a_i\opt \\ 1/2 & \text{if } a_{-i} = a_i\opt \\ 1 & \text{if } a_{-i} < a_i\opt \end{cases}\]Fix some $a_i \ne a_i\opt$. For any $a_{-i} \ne a\opt_{i}$, $v_i(a_i,a_{-i}) \le 1$. Furthermore, if $a_{-i} = a\opt_{i}$, then $v(a_i,a_{-i}) = 0$. Thus, $a_i\opt$ weakly dominates any other strategy. If $a_{-i} = a\opt_{i}$, then $v_i(a_i\opt,a_{-i}) = 1/2 > 0 = v_i(a_i',a_{-i})$ for any $a_i' \ne a_i\opt$, so this is a pure-strategy Nash equilibrium.
	\item Towards a contradiction, assume that there exists some other equilibrium $a'$ where $i$ plays $a_i' \ne a_i\opt$. WLOG, assume that $a_i' > a_i\opt$. If $a'_{-i} = a\opt$, then $i$ attains 0 always and could improve to 1/2 by switching to $a_i\opt$. Since $0$ and $1$ are strictly dominated, in any Nash equilibrium $a_{-i}' \in (0,1)$. If $a_{-i}' = a_i'$, then $i$ could improve by switching to $a_i\opt$ and getting 1 instead of 1/2. If $a_{-i}' < a_i'$, then $i$ could improve by switching to $a_i\opt$ and getting 1 instead of 0. If $a_{-i}' > a_i$, then $-i$ would improve by switching to $a_i\opt$ and attain 1 instead of 0. In all cases, this is not a Nash equilibrium because at least one player is incentivized to deviate. Note that this holds also with mixed strategies, considering expected payoff rather than actual payoff. Thus, there can be no other equilibria and $a\opt$ is the unique Nash equilibrium. 
\end{enumerate}

\subsection*{Problem 5}

\begin{enumerate}
	\item No. The strategy of voting for the opposing candidate is strictly dominated by not voting, since (i) in the case where the supporting candidate wins, they attain $1-c$ instead of 1, (ii) in the case where the opposing candidate wins, they attain $-c$ instead of 0, and (iii) in the case where they are pivotal, they attain $-c$ instead of 1. In every case, they get strictly lower payoff.
	\item We have to complete two tasks here: construct $q$ such that every member of $T_1$ is indifferent between voting and not voting, and construct a condition under which every member of $T_2$ is content to abstain. If $T_2$ never votes, a single vote will win (and zero votes will lose). Take some $i$ in $T_1$. The expected payoff from not voting is 1 times the probability that somebody else votes, plus zero times the probability that nobody else votes. The expected payoff from voting is $1-c$. We will set these equal to ensure a mixed strategy exists: \[u_i(\text{abstain}) = 1 - (1-q)^{M-1} = 1-c = u_i(\text{vote}) \Longrightarrow q = 1 - c^{1/(M-1)}\]Next, we check that a member $j \in T_2$ is willing to abstain under these conditions. They attain $1$ if nobody on $T_1$ votes, $0$ otherwise. For voting, they attain $1-c$ if exactly $0$ or $1$ members of $T_1$ vote, and $-c$ otherwise. For them to abstain, we need that \[u_j(\text{abstain}) \ge u_j(\text{vote}) \Longrightarrow (1-q)^M \ge [(1-q)^M + Mq(1-q)^{M-1}](1-c) -  \parl1-[(1-q)^M + Mq(1-q)^{M-1}]\parr c\]which simplifies to\[(1-q)^M \ge [(1-q)^M + Mq(1-q)^{M-1}]-c \Longrightarrow c \ge Mq(1-q)^{M-1}\]Substituting in the formula for $q$ from $T_1$'s condition, we get that this is equivalent to \[c \ge M(1-c^{1/(M-1)})c \Longrightarrow c \ge \parl 1 - \frac{1}{M}\parr^{M-1}\]Thus, as long as $c$ is large enough, there exists an equilibrium in which each member of $T_1$ votes with probability $q$ and each member of $T_2$ abstains. Putting this into practice with $c = 1/2$ and $M = 25$, we first check:\[\frac{1}{2} \ge \parl 1 - \frac{1}{25}\parr^{24} \approx 0.375\]So $T_2$ is willing to abstain. Then,\[q = 1 - \parl\frac{1}{2}\parr^{1/24} \approx 1 - 0.972 = 0.028 \]so each member of $T_1$ votes approximately $2.8\%$ of the time.
	\item Fix some $k \in (0,\min\{M-1,N\})$. Observe that if all members of $T_1$ except for $i$ vote with probability $q$, the number of other votes is $X \sim \text{binom}(q,M-1)$. We need $i$ to be indifferent between voting and not voting. We can characterize their expected utility as follows: \begin{align*} u(\text{abstain}) &= \prob\{X > k\} \cdot 1 + \prob\{X \le k\} \cdot 0 = \prob\{X \ge k+1\} \\ u(\text{vote}) &= \prob\{X \ge k\} \cdot (1-c) + \prob\{X < k\}\cdot(-c) = \prob\{X \ge k\} - c \end{align*}Setting them equal, we get \[\prob\{X \ge k+1\} = \prob\{X \ge k\} - c  \Longrightarrow c = \prob\{X = k\} = {M-1 \choose k}q^k(1-q)^{M-1-k}\]which defines $q$, if a solution to this equation in $(0,1)$ exists. To check whether the members of $T_2$ will adhere to their strategy, we will check the voters (among the $k$) and the abstainers (among the $N-k$). Define the number of $T_1$ votes as $Y \sim \text{binom}(q,M)$. For a voter, we need that the utility of voting is greater than the utility of abstaining, so\[\prob\{Y \le k\} \cdot (1-c) + \prob\{Y > k\} \cdot -c \ge \prob\{Y \le k-1\}\]which simplifies to\[\prob\{Y \le k\} - c\ge \prob\{Y \le k-1\} \Longrightarrow c \le \prob\{Y = k\}\]and since $Y$ is binomially distributed, this becomes \[c \le {M \choose k} q^k(1-q)^{M-k}\]For an abstainer, we need that the utility of abstaining is greater than the utility of voting, so \[\prob\{Y \le k\} \ge \prob\{Y\le k+1\}\cdot(1-c) + \prob\{Y > k+1\}\cdot(-c)\]which simplifies to \[c \ge \prob\{Y = k+1\} = {M \choose k+1} q^{k+1}(1-q)^{M - 1 - k}\]So the necessary and sufficient conditions for this equilibrium to exist are that:\[c \in \barl {M \choose k+1} q^{k+1}(1-q)^{M - 1 - k},{M \choose k} q^k(1-q)^{M-k}\barr\] and that there exists $q\in (0,1)$ that satisfies \[c = {M-1 \choose k}q^k(1-q)^{M-1-k}\]
	\item Fix some $c\in (0,1)$ and some $M,N \ge 2$. Assume that $k$ members of $T_2$ are voting. From the condition in part (3), there must exist $q$ such that \[c = {M-1 \choose k}q^k(1-q)^{M-1-k}\]By continuity, this determines a unique $q$ as long as \[c \le \max_{q \in (0,1)} \barl {M-1 \choose k}q^k(1-q)^{M-1-k}\barr\]Where the condition is met for any $k$ since the Binomial pdf can attain any value in $(0,1]$. Furthermore, assuming that the condition defining $q$ holds, it implies that \[c \le {M \choose k}q^k (1-q)^{M-k}\]so the only condition left on the $T_2$ voters is that \[c \ge {M \choose k+1} q^{k+1}(1-q)^{M - 1 - k}\]and combining with the condition defining $q$, this simplifies to \[q \le \frac{k+1}{M}\]Thus, if $N \ge M-1$, this condition can be satisfied trivially by choosing $k = M-1$. It remains to show that this holds when $k = N < M-1$. The $T_1$ condition admits a unique $q$, as always, and we need to check that $q \le \frac{N+1}{M}$. We have that \[c = {M-1 \choose N}q^N(1-q)^{M-N-1}\]So by the properties of binomial pdfs, the unique maximum of $q$ here is $q = \frac{N}{M-1} \le \frac{N+1}{M}$. Thus, the unique $q$ will satisfy this condition, and there will always exist (at least one) equilibrium for any $c \in (0,1)$ in which $k \in \{M-1,N\}$ members of $T_2$ vote and there exists a unique $q$ such that each member of $T_1$ votes with probability $q$.
\end{enumerate}










\end{document}