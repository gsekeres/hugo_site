\documentclass[10pt]{article}

\input{/Users/gabesekeres/Dropbox/LaTeX_Docs/pset_preamble.tex}

\course{ECON 6100}
\pset{3}
\begin{document}
\maketitle

\begin{enumerate}
	\item We know that a matrix $A$ is productive if and only if the $(I-A)^{-1}$ has non-negative columns and is non-singular. We have that \[I-A_1 = \matrixc{0.4 & -0.2 & -0.1 \\ -0.3 & 0.8 & -0.4\\ -0.2 & -0.4 & 0.7} \text{ and so } (I-A_1)^{-1} \approx \matrixc{5.405 & 2.432 & 2.162\\3.919 & 3.514 & 2.568 \\ 3.784 & 2.703 & 3.514} \]which has strictly positive columns and has full rank, so the $A_1$ is productive. We also have that \[I - A_2 = \matrixc{0.4 & -0.5 \\ -0.1 & 0.5} \text{ and so } (I - A_2)^{-1} = \matrixc{10 / 3 & 10 / 3 \\ 2/3 & 8/3}\]which has strictly positive columns and is full rank, so $A_2$ is productive.
	\item \begin{proof} ($\Rightarrow$) We have that $A$ is productive, meaning that there exists $x\opt \gg 0$ such that $x\opt \gg Ax\opt$. Note that $Ix\opt \gg 0$, so $(I-A)^{-1}(I-A)x\opt \gg 0$, and since $(I-A)x\opt \gg 0$ since $x\opt \gg Ax\opt$, we have that $(I-A)^{-1} \gg 0$. \\\\($\Leftarrow$) We have that $(I-A)^{-1}$ has non-negative columns, $(I-A)^{-1}x \ge 0$ for any $x \ge 0$. Define $x\opt = (I-A)^{-1}e_j$ for some index $j$, and since $e_j \ge 0$, $x\opt \ge 0$. Further, $(I-A)x\opt = e_j \Longrightarrow x\opt \gg Ax\opt$, so $A$ is productive.\end{proof}
	\item \begin{proof} FSOC, assume that every column sum of $A \in \reals^{n \times n}$ is greater than 1. Then taking the system of equations required for $A$ to be productive (for the assumed $x\opt \in \reals^n_+$), we have that \begin{align*} a_{11} x_1 + a_{12} x_2 + \cdots + a_{1n} x_n &< x_1 \\a_{21} x_1 + a_{22} x_2 + \cdots + a_{2n} x_n &< x_2  \\ &\vdots \\a_{n1} x_1 + a_{n2}x_2 + \cdots + a_{nn} x_n &< x_n \end{align*}Summing each equation, this becomes \[\sum_{i=1}^n a_{i1}x_1 + \sum_{i=1}^n a_{i2}x_2 + \cdots + \sum_{i=1}^n a_{in}x_n < x_1 + x_2 + \cdots + x_n\]and since $\sum_{i=1}^n a_{ij} > 1$ for every $j$, this implies that $Ax \not\ll x$, contradicting the assumption that $A$ is productive. \end{proof}
	\item We need a price vector $(p_0,p) \in \reals_+ \times \reals^3_+$ such that the profit matrix $\pi$ is the zero matrix. Formally, we want \[p \cdot (I-A) - a_0 = 0 \Longrightarrow a_0 = p\cdot(I-A) \Longrightarrow \matrixc{1 \\ 1 \\ 1} = \matrixc{p_1 \\ p_2 \\ p_3} \matrixc{0.9 & -0.4 & -0.3 \\ -0.2 & 0.3 & 0 \\ -0.1 & -0.1 & 0.5}\]so the equilibrium prices are the solution to the following system of equations: \begin{align*} 0.9p_1 - 0.4p_2 - 0.3p_3 &= 1 \\ -0.2p_1 + 0.3p_2 &= 1 \\ -0.1p_1 - 0.1p_2 + 0.5p_3 &= 1 \\\Longrightarrow (p_1,p_2,p_3) &= (5.875,7.25,4.625) \end{align*}where we are assuming $p_0 = 1$ to guarantee a unique solution.
	\item Irreducible matrices. \begin{enumerate} \item \begin{proof} ($\Rightarrow$) Assume we have an irreducible square matrix $A$. This means that the graph is strongly connected. Take some $i,j$. Since the graph is strongly connected, there is a path of length $m$ from $i$ to $j$. If the path is of length $1$, then $a_{ij} > 0$. If the path is of length 2, then there exists $k$ such that $a_{ik} > 0$ and $a_{kj} > 0$, so $A^2_{ij} = a_{i1} \cdot a_{1j} + \cdots + a_{ik} \cdot a_{kj} + \cdots > 0$. If the path is of length $m$, then there exist $m-1$ intermediate points (indexed $n_1,\dots,n_{m-1}$ such that $a_{in_1} > 0$, $a_{n_xn_{x+1}} > 0$, and $a_{n_{m-1}j} > 0$ for all $x = 1,\dots,m-2$. Thus, $A^m_{ij} = \cdots + a_{in_1} \cdot a_{n_1n_2} \cdot \cdots a_{n_{m-1}j} + \cdots > 0$. \\\\ ($\Leftarrow$) Assume that for each $i,j$ there exists $m$ such that $A^m_{ij} > 0$. This means that there is at least one product in the sum $A^m_{ij}$ that is strictly positive. As with all products, it will begin with $a_{ix}$ for some $x$ and end with $a_{yj}$ for some $j$. All of the intermediate terms connecting $x$ to $y$ will be strictly positive, as will $a_{ix}$ and $a_{yj}$. Thus, they constitute a connected path from $i$ to $j$, so the graph is strongly connected and the matrix is irreducible. \end{proof} \item \begin{proof} Assume that $a_0 > 0$ but $a_0 \not\gg 0$, meaning that for at least one $i$, $a_{0i} = 0$, and further assume that $A$ is irreducible. Recall that per-unit profits are $\pi = p(I-A) - a_0$. Take $p = (I-A)^{-1}a_0$, so $\pi =0$. It remains to show that $p$ is strictly positive, for equilibrium to exist. Recall that $a_0$ has at least one non-zero element. Since $(I-A)^{-1} = I + A + A^2 + \cdots + A^n + \cdots$, and since $A$ is irreducible, there exists $n$ sufficiently large that all elements of $A$ are strictly positive. Thus, $(I-A)^{-1}$ has all elements strictly positive, so $p$ is strictly positive by $a_0 > 0$.\end{proof}\end{enumerate}
\end{enumerate}














\end{document}