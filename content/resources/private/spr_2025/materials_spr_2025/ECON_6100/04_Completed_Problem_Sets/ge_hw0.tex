\documentclass[10pt]{article}

\input{/Users/gabesekeres/Dropbox/LaTeX_Docs/pset_preamble.tex}

\course{ECON 6100}
\pset{0}
\begin{document}
\maketitle

\begin{enumerate}
	\item \begin{proof}\footnote{Partially from Patrick, who had a really nice proof in the solutions to the 6170 final. His second part is way cleaner than mine was, I ended up proving strong and strict separation, for no real reason.} Define $\mathcal{H} = \{ H : C \subseteq H, H$ is a closed half-space$\}$. We want to show that $C = \bigcap_{H \in \mathcal{H}}H$. We will do so with double set containment. 
	
	First, $C \subseteq \bigcap_{H \in \mathcal{H}}H$ simply -- for any $c \in C$, since $C \subseteq H \forall H \in \mathcal{H}$, $c \in H \forall H \in \mathcal{H}$, so $c \in \bigcap_{H \in \mathcal{H}}H$. Thus, $C \subseteq \bigcap_{H \in \mathcal{H}}H$.
	
	Next, $\bigcap_{H \in \mathcal{H}}H \subseteq C$. We will show by contrapositive. Take some $x \not\in C$. Then since singleton sets are closed and convex, since $C$ is closed and convex by assumption, and since $\{x\} \cap C = \emptyset$, we have that the Strong Separating Hyperplane Theorem applies, meaning that there exists a hyperplane $P \ne 0$ which strongly separates $\{x\}$ and $C$. Since this is strong separation, $x \not\in P$, meaning that the (weak) halfspace generated by $P$ that contains $C$ does not contain $x$. Since $P \in \mathcal{H}$, we have that $x \not\in \bigcap_{H \in \mathcal{H}}H$.
	
	Since we have that $C \subseteq \bigcap_{H \in \mathcal{H}}H$ and $\bigcap_{H \in \mathcal{H}}H \subseteq C$, we have that $C = \bigcap_{H \in \mathcal{H}}H$.
	\end{proof}
	\item We have the concave support function of $C$, $e_C(p)=\inf\{p\cdot x : x \in C\}$.\begin{enumerate}\item \begin{proof}Fix some $p,p' \in \reals^n$ and some $\alpha \in [0,1]$, and define $p'' = \alpha p + (1-\alpha)p'$. If either $e_C(p)$ or $e_C(p')$ are equal to $-\infty$, then trivially $e_C(p'') \ge -\infty = \alpha e_C(p) + (1-\alpha)e_C(p')$. If either $e_C(p)$ or $e_C(p')$ are equal to $\infty$, then $C = \emptyset$ and so $e_C(p'') = \infty \ge \alpha e_C(p) + (1-\alpha)e_C(p')$. From here, assume that $e_C(p),e_C(p')$ are finite. Define $x,x',x'' \in C$ where $p \cdot x = e_C(p)$, $p'\cdot x' = e_C(p')$, and $p''\cdot x''=e_C(p'')$. Existence follows from closed and a finite infimum, meaning that extrema are attained. (Uniqueness isn't necessarily true, but not necessary here). We have that \[e_C(p'') = p''\cdot x'' = \alpha p \cdot x'' + (1-\alpha)p'\cdot x'' \ge \alpha p \cdot x + (1-\alpha)p'\cdot x' = \alpha e_C(p) + (1-\alpha)e_C(p')\]where the inequality follows from the attained infimum. Thus, $e_C(\cdot)$ is concave.\end{proof} \item \begin{proof}Fix some $p \in \reals^n$ and some $\lambda \in \reals_{++}$. If $e_C(p) = \infty$, then $C = \emptyset$ so $e_C(\lambda p) = \infty \equiv \lambda \infty$. If $e_C(p) = -\infty$, then there exists a sequence $\{x_n\} \in C$ such that $\lim_{n\to\infty} p \cdot x_n = -\infty$, so $\lim_{n\to\infty} \lambda p \cdot x_n = \lambda \lim_{n\to\infty} p \cdot x_n = \lambda \cdot (-\infty)$, From here, assume that $e_C(p)$ is finite. Then since $C$ is closed, there exists $x \in C$ such that $p \cdot x = e_C(p)$. It follows directly that $e_C(\lambda p) \le \lambda p \cdot x = \lambda e_C(p)$. It remains to show that $\lambda e_C(p) \le e_C(\lambda p)$. FSOC, assume that there exists $x' \in C$ such that $\lambda p \cdot x' < \lambda e_C(p)$. Then we would have that $p \cdot  x' < p \cdot x$, contradicting the definition of $e_C(p)$ as the minimum. Thus, $e_C(\lambda p) = \lambda e_C(p)$.\end{proof} \item If $e_C(p) = -\infty$, then $C$ is unbounded in at least one dimension $i$, specifically in the opposite direction of $p_i$, where $p_i$ is nonzero. In this dimension, there exists a sequence $\{x_n\} \in C$ such that the $i$th coordinate of $x$ diverges, so that $\lim_{n\to\infty} p \cdot x_n = -\infty$. \item \begin{proof} First, assume that the halfspace $ [p \ge \alpha] \subseteq \reals^n$ contains $C$. Then for all $x \in C$, $p \cdot x \ge \alpha$, meaning that $\alpha \le \inf\{p \cdot x : x \in C\} = e_C(p)$. Next, assume that $\alpha \le e_C(p)$. Then either $e_C(p) = \infty$, meaning that $C$ is empty and contained in any nonempty set, including the halfspace, or $e_C(p)$ is finite, so there exists $x \in C$ such that $p \cdot x = e_C(p)$. Since $\alpha \le p \cdot x$, from the definition of extrema $\alpha \le p \cdot y \forall y \in C$, so $C \subseteq [p \ge \alpha]$. \end{proof} \end{enumerate}
	\item \begin{proof} Assume that $f$ is concave. Take some $(x,y)$ and $(x',y')$ such that $y \le f(x)$ and $y' \le f(x')$, and fix $\alpha \in [0,1]$. Then we have that since $f$ is concave,\[\alpha y + (1-\alpha)y' \le \alpha f(x) + (1-\alpha)f(x') \le f(\alpha x + (1-\alpha)x')\] so $\alpha (x,y) + (1-\alpha) (x',y')$ is in the subgraph of $f$, and it is convex. Next, assume that the subgraph of $f$ is convex. Take some $x,x' \in \reals^n$, and fix $\alpha \in [0,1]$. Set $y = f(x)$ and $y' = f(x')$. Since $y\le f(x)$ and $y' \le f(x')$, $(x,y)$ and $(x',y')$ are both in the subgraph. Then we have that since the subgraph is convex, $\alpha y + (1-\alpha)y' \le f(\alpha x + (1-\alpha)x')$, meaning that $f(\alpha x + (1-\alpha)x') \ge \alpha f(x) + (1-\alpha)f(x')$, so $f$ is concave.\end{proof}
	\item Two simple examples: $X = [0,1] = Y$, which are both closed and convex and cannot be separated as they are the same set. Or, $X$ is nonempty closed and convex and $Y = \emptyset$, which is vacuously closed and convex. These cannot be separated since each hyperplane (again, vacuously) contains $Y$.
	\item \begin{proof} If $yA \ll 0$, then $yAx < 0$ for any $x > 0$, meaning that $Ax \ne 0$ for any $x > 0$. On the other side, if $Ax = 0$ for some $x > 0$, taking some element $i$ of $x$ where $x_i > 0$, we have that the $i$th column of $A$ must necessarily be 0, for complimentary slackness. Thus, the $i$th element of $yA$ must be 0 for any $y$, so $yA \ll 0$ has no solutions. Thus, the two results are mutually exclusive. \end{proof}
\end{enumerate}




















\end{document}