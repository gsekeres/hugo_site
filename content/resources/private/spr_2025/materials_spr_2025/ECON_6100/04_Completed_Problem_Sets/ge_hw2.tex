\documentclass[10pt]{article}

\input{/Users/gabesekeres/Dropbox/LaTeX_Docs/pset_preamble.tex}

\course{ECON 6100}
\pset{2}
\begin{document}
\maketitle

\begin{enumerate}
	\item We have a two-player zero-sum game described by a matrix $M$. \begin{enumerate} \item Assume that $M \in \reals^{m \times n}$, so $p \in \Delta_m \subseteq \reals^m$, the $m$-simplex. Then we have that the expected payoff of choosing mixed strategy $p$ when the column player chooses $j$ with probability 1 is \[p_1 \cdot m_{1j} + p_2\cdot m_{2j} + \cdots + p_m \cdot m_{mj} = p \cdot m^{j}\]where $m^j$ is the $j$th column of the matrix $M$. The vector of possible payoffs is directly $m^j$. \item The row player is working to maximize their security level, the most utility that they can guarantee themselves, regardless of what the column player chooses. We can represent their problem as \begin{align*}
		 \max&\; z \\ \st \quad \sum_{i=1}^m p_i &= 1 \\ p_i &\ge 0 \forall i = 1,\dots,m \\ \sum_{i=1}^m m_{ij} p_i - z &\ge 0 \forall j = 1,\dots,n \end{align*}In canonical form, remembering that $z$ can be negative, this becomes\begin{align*} \underbrace{(z\opt,p\opt)}_{\in \reals \times \reals^m} = \max &\; 1 \cdot z^+ - 1 \cdot z^- + 0 \cdot p_1 + \cdots + 0 \cdot p_m \\ \st \quad \sum_{i=1}^m p_i &\le 1 \\ -\sum_{i=1}^m p_i &\le -1 \\ z^+ - z^- - \sum_{i=1}^m m_{ij}p_i &\le 0 \quad \forall j = 1,\dots,n\\ z^+,z^-,p &\ge 0\end{align*}where the introduced notation denotes $z\opt = z^+ - z^-$ as the attained maximum of the problem, and $p\opt$ as the maximizing strategy. The matrices for this are \[A=\matrixc{0&0 & 1 & \cdots & 1 \\ 0&0 & -1 & \cdots & -1 \\ 1 &-1& -m_{11} & \cdots & -m_{m1} \\ 1 &-1& -m_{12} & \cdots & -m_{m2} \\ \vdots &\vdots & \vdots & \ddots & \vdots \\ 1 &-1& -m_{1n} & \cdots & -m_{mn}} \qquad c = \matrixc{1\\-1 \\ 0 \\ \vdots \\ 0}^T \qquad b = \matrixc{1 \\ -1 \\ 0 \\ \vdots \\ 0} \]\item Converting our canonical form directly to the dual problem, we get that the column player's strategy is defined by \begin{align*} (w\opt,q\opt) = \min&\; 1 \cdot w^+ - 1 \cdot w^- + 0 \cdot q_1 + \cdots + 0 \cdot q_n \\ \st \quad \sum_{j=1}^n q_j &\ge 1\\ -\sum_{j=1}^n q_j &\ge -1\\ w^+ - w^- - \sum_{j=1}^n m_{ij}q_j &\ge 0 \quad \forall i = 1,\dots,m \\ w^+,w^-,q &\ge 0 \end{align*}where $w\opt = w^+ - w^-$ can be negative, and $q\opt$ is the maximizing choice of strategy. \item We have directly that $p\opt$ maximizes row's security value. It remains to show that $q\opt$ maximizes column's security value. From the structure of the dual, we can see that $q\opt$ is minimizing the expected payoffs to row of a strategy, subject to that strategy being optimal for row. Recall that a payoff of $m_{ij}$ to row leads to a payoff of $-m_{ij}$ to column, so this problem could be represented as a maximization problem by multiplying $w$ and $m_{ij}$ by $-1$: \begin{align*} \max&\; 1 \cdot -w^+ + 1 \cdot w^- + 0 \cdot q_1 + \cdots + 0 \cdot q_n \\ \st \quad \sum_{j=1}^n q_j &\ge 1\\ -\sum_{j=1}^n q_j &\ge -1\\ -w^+ + w^- + \sum_{j=1}^n m_{ij}q_j &\ge 0 \quad \forall i = 1,\dots,m \\ w^+,w^-,q &\ge 0 \end{align*} and by multiplying the main constraint entirely by $-1$ also, we can precisely recover the primal linear program from part (b). Thus, $q\opt$ maximizes column's security value. \item Intuitively, a Nash equilibrium strategy maximizes payoff for the player in question, subject to their opponent(s) response, where we assume that they are best-responding. This solution concept replicates that, but where each player assumes that their opponent's best response is to maximize their security value. It's straightforward to see that in a two-player zero-sum game, where a positive payoff to row leads to a negative payoff of the same amount to column, the players are directly competing, and maximizing their own payoff (\ie\;best-responding) is the same as minimizing their opponent's payoff. In that case, maximizing the security value is precisely equivalent to best-responding, so in two-player zero-sum games the $(p\opt,q\opt)$ solution concept is exactly the same as the Nash equilibrium solution concept. \item With this solution concept, we have all of the properties to solutions to linear programming problems. This leads to additional structure over the solution set to Nash equilibria in a given game. For one, the solutions to two-player zero-sum games are unique, which is (obviously) not the case for Nash equilibria. For another, from one player's solution we can find the other's by using the dual results expressed above. In more general finite games, it is not necessarily true that one player's strategy can be recovered from another's. Finally, since the solutions to two-player zero-sum games are linear programs, they can be found in polynomial time (specifically, matrix multiplication time -- the current best algorithm belongs to complexity class $\tilde{O}((n^\omega + n^{2.5 - \alpha / 2} + n^{2+1/6})L)$), while finding the Nash equilibrium to a general finite game is a PPAD-complete problem, a complexity class similar to NP completeness.\end{enumerate}
\end{enumerate}




























































\end{document}