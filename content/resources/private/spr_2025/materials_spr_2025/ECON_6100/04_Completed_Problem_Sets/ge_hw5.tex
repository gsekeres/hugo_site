\documentclass[10pt]{article}

\input{/Users/gabesekeres/Dropbox/LaTeX_Docs/pset_preamble.tex}

\course{ECON 6100}
\pset{5}
\begin{document}
\maketitle

\begin{enumerate}
	\item Suppose that we have $N$ plots of land, each described by some $\{a_n,c_n\}$ where $a_n$ denotes the output of a single unit of labor and $c_n$ the total number of workers who can produce on the land.\footnote{Note the difference from the way Larry described it in class -- I'm having $a_n$ be output per worker rather than worker per output. It's, to me, more intuitive this way, so that an increase in either $a_n$ or $c_n$ increases the amount of output the land can provide.} We can think of $a_n$ as `soil quality' and $c_n$ as `size'. Formally, assume that $a_n \sim F(\cdot)$ and $c_n \sim G(\cdot)$ for arbitrary and unknown distributions $F$ and $G$, and assume that they are independently drawn. We have $L$ workers in the entire market, where $\ell_n$ are assigned to plot $n$ and $e$ are assigned to some subsistence activity with (exogenous) marginal product $s < 1$. We have a wage rate $w$ and an output price for the good of 1. Each land has an (endogenous) rental price $r_n$, assessed per worker on each plot of land. \\\\Formally, equilibrium will consist of finding the competitive wage and rent prices, as well as the number of workers on each plot and how many remain in subsistence. We look for a tuple $\langle w, \{r_n\},\{\ell_n\},e\rangle$. Imagine a social planner is attempting to maximize output subject to the previous conditions. They would solve the linear program\[\max_{\{\ell_n\},e} \sum_{n=1}^N a_n \ell_n + se\]subject to\begin{align*} \sum_{n=1}^N \ell_n + e &= L \\ \ell_n &\le c_n &&\forall n = 1,\dots,N\\ \ell_n,e &\ge 0 &&\forall n = 1,\dots,N\end{align*}The dual of this linear program is thus\[\min_{w,\{r_n\}} Lw + \sum_{n=1}^N c_n r_n \]subject to\begin{align*} w + r_n &\ge a_n &&\forall n = 1,\dots,N \\ w &\ge s \\ r_n &\ge 0 &&\forall n = 1,\dots,N\end{align*}where $r_n$ is the shadow price on the capacity constraint. From complementary slackness, as long as $\ell_n < c_n$ for some plot $n$, then $r_n = 0$. Otherwise, we will have that $w + r_n= a_n$, so the worker pays the entirety of the difference between their wage and their marginal productivity on a certain plot in rent. The solutions to these paired problems will be exactly our equilibrium conditions -- the primal will give us the assignment of workers to plots, and the dual will give us the wage and rental prices. Intuitively, we get exactly the assignment that Ricardo predicted. If we sort the plots such that $a_1 \ge a_2 \ge \cdots \ge a_N$, then we will assign $\ell_1 = c_1$ workers to plot 1, $\ell_2 = c_2$ workers to plot 2, and so on, until we have assigned $L$ total workers. The marginal plot $i$, where we entirely fill $i-1$ and assign no workers to $i+1$, will have some output $a_i$. We will set $w = a_i$, so that in the marginal plot we charge $r_i = 0$. We charge $r_n = 0$ for $n \ge i$, and $r_n > 0$ for $n < i$. Observe that if $a_i < s$, we instead set $w = s$, and only assign workers to the plots $j$ that have $a_j \ge s$. All other workers (a positive number) will work the subsistence activity. Everything depends on what the productivity of the marginal plot $i$ is.
	\item Introduce now an additional element to the description of the plots of land, $b_n$ which describes the productivity of a single unit of capital in enhancing the output. We now have that each of the plots is described by $\{a_n,b_n,c_n,d_n\}$, where $a_n$ and $c_n$ are as above, $b_n \sim H(\cdot)$ for some distribution $H$, and $d_n$ is the capacity constraint for capital. If we have $K$ total capital, where $k_n$ are assigned to plot $n$, and $K - \sum_{i=1}^N k_n$ goes unused. We have a capital rental rate $q > 0$. The maximization linear program is now \[\max_{\{\ell_n,k_n\},e} \sum_{n=1}^N (a_n \ell_n + b_nk_n) + se\]subject to \begin{align*} \sum_{n=1}^N \ell_n + e &= L \\ \sum_{n=1}^N k_n &\le K \\ \ell_n &\le c_n &&\forall n=1,\dots,N \\ k_n &\le d_n &&\forall n=1,\dots,N \\ \ell_n,k_n,e &\ge 0 &&\forall n = 1,\dots,N\end{align*}The dual is now \[\min_{w,q,\{r_n,t_n\}} Lw + Kq + \sum_{n=1}^N (c_nr_n + d_nt_n)\]subject to\begin{align*} w + r_n &\ge a_n &&\forall n = 1,\dots,N \\ q + d_n &\ge b_n &&\forall n = 1,\dots,N \\w &\ge s \\r_n,t_n,q &\ge 0 &&\forall n = 1,\dots,N\end{align*} Like above, workers are assigned to the most productive plots, and capital is assigned similarly. 
	\item Wages in my model have no effect on the return to capital. Instead, they are determined endogenously by the productivity of the land. If $a_n$ and $b_n$ are correlated in some way (which is not unreasonable), we might see that wages rising actually increases the return to capital. Alternatively, the other way we could write this model is to be nonlinear in capital, where we have that the production function is something akin to $a_n\ell_n \cdot b_nk_n$. In that way, an increase in the wage would lead to a smoothing of the amount of capital used on the most productive plots, which would decrease the return to capital. 
	\item As we talked about in part (1), as the population grows, if nobody is using the subsistence activity the number of plots in use will grow and the wage will fall. On the other hand, if anyone is participating in the subsistence activity, population growth will go only to the subsistence activity and will have no effect on the wage level.
\end{enumerate}






\end{document}